\documentclass[../HiganMain]{subfiles}

\setcounter{chapter}{0}

\begin{document}
\Entry{捨子花}{\tt Second Encounter:Falling in the sky}

吐き気がまた襲ってきた。私の立つこの大地、いや、地球そのものがシーソーに乗っけられて、
ぐわんぐわんと揺らされている、そんな感覚。
だから私は、その先で無邪気に飛び跳ねる元凶を叩き潰してやりたかったが、残念、時間切れだ。

「また『発作』か?今度はどんな死に方をしたんだ?私は」\\
私の雇用主、\ruby{末巫冬姫}{ミカナギカズキ}は、
学校の先生のように、詳細なレポートを要求する。\\
「首吊り。ちょうどそこにあるドアノブみたいなのに縄を括って、胡座をかいてた。
天井には体重を支えられるものはなかったみたい」\\
「自殺か?思い当たる節はないが\scalebox{3}[1]{―}」\\
「もしかしたら別の可能性……窒息プレイとか」\\
「私はそんな悪趣味な人間じゃないよ」\\
なんて悪趣味な人間が宣う。

普通の人間に、この空気の漂う部屋は似合わない。
喫茶店をそのままに利用したこの事務所は、本来シックな雰囲気を持つべきなのだが、
完全にそれを放棄している。私の第一印象は、どこかの女性誌にでも乗っていそうな、
おしゃれな建物だったのに、今では魔女の館だ。
怪しげな洋書や和綴の古書が無造作に積み上げられ、
その中から選抜されたものだけが、本棚という安寧を与えられている。
もっとも、この配置は彼女の気まぐれで、同じ配列は一週間も保たない。
だがここまでは単なる、ずぼらな読書家の凡例だろう。
問題は、もっと根源的なところに根付いている。
端的に言って、ここは呪われているんだ。
まずはじめに、前オーナーはここで死んだ。
債務に押し潰されたのだ。
二つ目に、彼女が住み着いたこと。
そして最後に、私がここにいること。
詳細はいずれ。
ここにいると、記憶が曖昧になりやすいから。

\scalebox{3}[1]{―}私の頭の悪い冗談を笑って済ませた後、
数枚の紙を整理したところで、彼女は私の予言への\ruby{緒}{いとぐち}を見つけてしまったようだ。\\
「ああ\scalebox{3}[1]{―}これは、不味いな」\\
「なにそれ」\\
「通帳のコピー」\\
\scalebox{3}[1]{―}用があって印刷したのだという。
だったら、前もって知っているはずだろうに。
私は指摘しようとしたけれど、彼女の数字嫌いを思い出して、口をつぐんだ。
彼女にとっては、百も一万も対して変わらない。だってゼロが二つから四つに変わっただけだから。
遠目で見れば、どっちも点だ。

欠落した金銭感覚は、彼女の数多くの欠点の一つ。
私は、反面教師にしないと。
そのけがあることは自覚している。

見れば残高の額は、小学生のお小遣いみたいなものだった。
いわゆる破綻寸前。
ちょっとガスと電気と水道が使えなくなるぐらい、
魔術師にとってはどうでもいいこと。
と、以前豪語していたが、実際危機を目の前にした彼女は、目に見えて焦っていた。
個人の探偵事務所なんて、安定した収入は端から期待できない職種だろうに、
そのへんの危機感がやはり金銭感覚とともに、彼女から欠落していた。
まあでも、確かに出来るのかもしれない。

雨水をサイダーにぐらい、簡単だろう。\\

暗い顔で絶望する冬姫を置いて事務所を出たあと、
私は散歩がてらに懐かしの母校に立ち寄った。

時刻はもう午後四時を過ぎ、チラホラと下校する中学生たちの姿が見える。
夕暮れは各々を赤く染めて、紅い、もっとも古い光に人間たちは時の流れを背負って帰路につく。
気怠い顔に、はしゃぐ集団に、無表情にただ前へと進む諸々。
私はその中から、ただ一人を見分けなければならない。
しかも、これが厄介なことに彼の雰囲気はすこぶる平凡で、まあ、それが普通なのだが、
人混みの中ではすぐに自己主張をしなくなる。
これが少し反抗的で、なおかつ非行な少年だったら、
髪の毛は不自然な茶色に塗れて、多少なりとも見つけやすくなるのだが、
いたって根は真面目なので、そもそも彼には染髪という発想がない。
だからこの似たような身長の波から、校則によって規定された画一的な髪型の陳列から、
彼を見極める眼光は、自然と険しく、人払いのそれになってしまうのだ。

校庭を囲む垣根を回って、たらたら漏れる脈の流出元をたどって見れば、
全ては一日の始まりに彼らが入る、化物の\ruby{腸}{はらわた}直行便の改札口に行き着く。
自分自身にあまり、中学生時代のいい思い出はないが、かと言って恨みつらみの類もない。
何もない日々に感じたことは少なく、ただ今でも覚えているぐらい単純な事柄が一つ。
あそこは人の心もわかり合えない怪物達の溜まり場。
社会からの体のいい隔離施設だということ。
だったら、怪獣の腹の中がお似合いだろう。
いつかヨナのように改心して、人々のために働ければいいのだが。
……己を鑑みれば、無理な話だろう。

さて、いらない考えはここで捨てよう。
私は哲学者でも思想家でも、ソフィストでもない。
ただ少し頭の悪い、この世界でたった一人の人間。
他人とは違う、重複のない私だけの私。
だから険しい顔はやめて、外交用の笑顔を作る。
一つの体に付着する信用は、いとも簡単に剥がれやすいものだからだ。
たとえ無関係と断じて無視しても、いずれそれは自分に帰ってくる。
だったら、無理にでも口角を上げておくべきなのは、誰だって理解できる鉄則だろう。

なぜなら彼は、ちょうど友達と一緒だったからだ。\\

友人達とはすぐに別れることになった。
信号を渡って、私達二人は右に、彼らは左に曲がったからだ。
三人組は大きく手を振って、わかりやすい媚びを私に売る。\\
「お姉ちゃんのこと好きみたい」\\
彼は言った。\\
「ガキのくせに、おませさんな奴らだな」\\
口調がつい男性的になる。たまにこうなるんだ。
大抵の場合、性に合わないことを言ってしまったとき、
語尾を変形させてなんとか不時着しようとするのだが、
これはその時の緩衝材みたいなもの。\\
「しょうがないよ。だって、綺麗なんだもん」\\
「何?なんか言った」\\
「なんにも」\\
白状すると聞こえていたが、私は聞き流すふりをした。
恥ずかしさと少々の驚き。まだ中学生だが、もう中学生だ。
思春期は近いし、異性を意識するのも当たり前か。
だからって、もう何年も付き合いのある人間に対して、そんな感情を抱くものなのだろうか。
羞恥に縮む声は、聞く方もこそばゆい。共感性羞恥。
ごまかしは大声で。\\
「そういえばさあ、小鳥はどうなの」\\
「何が」\\
「恋人」\\
「そ、そんなの」\\



\section*{就寝}
月明かりだけが、私の頼りだった。
電球では眩しすぎる私の目には、あの大きな照明の、しかし淡い光がちょうど良かった。
そしてここではどうしてか、月が大きく見える。
現実にはありえないほどに拡張された青白い月が、町を押しつぶそうとしている。
まるで私の合図を待っているみたいに。
もしかしてあの月は、中空でピアノ線に固定されていて、
後はほんの一突きでもすれば、私の細長く心もとない指でも墜ちていくだろう。
それは\scalebox{3}[1]{―}ほんのちっぽけな自尊心だ。

注意を切り替える。拘束されたように動こうとしない足に、
今日もまた落胆して、私は区切りにならない眠りに入ろうとする。
眠たくもない目を閉じることは、想像するよりずっと難しい。
だからまた目が泳いでしまう。ガラスの向こう側、真っ暗で光の積もる町の俯瞰へと。

のっぺりと張り付く壁紙のような風景は、一目でココがどれだけ異質な場所なのかを教えてくれる。
けれど、そこは決しておどろおどろしい魔境というわけでもなかった。
例えるならそう、子供部屋のような雰囲気。
ありとあらゆる『体験』に飢えている子供達を、満足させるために誂えられた世界。
延々と広がる文明の平野は好奇心を掻き立て、果てのない夜空は恐怖を覚えさせてくれる。
そして人工の光に飲まれながらも、懸命にぽつりぽつりと輝く星達は、
理由のない希望を与えて、安らかな眠りを誘う。

そんな景色を一望する丘の上に立つ、大きな洋風の屋敷が私の住処だった。

私の部屋は壁の一面がガラス張りになっていて、
綺麗な夜景をその隅々まで見渡せるようになっている。

だからここを『展望台』と街の人たちは呼んでいる。

周りは代わり映えのない規範的な住居ばかりのこの街に、古くからある異質な館。
町の住人たちにとって、ある種のシンボルとなるのも理解できなくはなかった。

だけど、そんなことはもう私には関係のないことだった。
日に日に弱っていく体は、ついに立ち上がる自由さえ私から奪い取ったのだから。
そんな体で町に下りることなど出来るはずもない。
結果、外との関わりなど気にする必要性もなく、この家がなんと言われようと関係はない。
死の瀬戸際を彷徨う私にはただ、この風景を睨み続けることしか出来なかった。

だから私はあれが恨めしい。
私を置いてけぼりにして勝手に前へ進んで、
何もかも有耶無耶にして平穏を歩み続ける彼等。

でもそれが私自身の逆恨みであることは明白で、
無意味な憎悪に限りある生命を費やすのは、あまりにも愚かだ。

しかし未練を捨てきれない、過去と折り合えない自分がいることも真実だった。

地主だったお父様が建てて、お金以外に残していった唯一の遺産であるこの住処。
『意味のある』という意味で遺された、私のヨスガであり足枷。
この束縛から解放されて、あの町で遊べたらどれだけ素敵なことだろうか。
日焼けに肌を掻いて、沁みる塩水に目を目を窄めることの、どれだけ不可能なことか。
願わくばもう一度\scalebox{3}[1]{―}いや、これはやめよう。\\

「お嬢様。まだ起きていらっしゃったのですか。\scalebox{3}[1]{―}お体を大切にしないと」\\
控えめに開いた扉の先から、いつもの声が聞こえる。
心配性な母親じみた声。\\
「うん、ごめんなさい。そうするわ」\\
これ以上彼女に無駄な時間を使わせたくはない。
私は掛け布団をガサガサ鳴らして、大げさに包まった。

おやすみなさい。

浮きかけた足を落として、私はベッドに埋没した。

\section{}
夏の雨ほど焦れったい物は無いと思う。
ついさっきまで晴れていた空から、急に降りだした雨。
しかも小雨なんて生易しい物じゃない。
それはもう台風でも来たんじゃないかって錯覚するぐらいのどしゃ降り。
最初は通り雨だと見くびっていた人々も、今は雨に追われて駆け回っている。
そんな私たちをあざ笑うかのように、温い豪雨は瞬く間に街を覆っていく。
行き場を失いアスファルトを右往左往する水を踏んで、
ぐしょぐしょになってしまったスニーカーや靴下は重く、そして纏わりついて気色が悪い。
いっそのこと脱いでしまいたいが、小学生でもあるまいし公道で裸足になることは流石に無理だ。
もちろん、傘など持ってないのだから上もびしょ濡れで、
まあこれはこれとしてちょっとした画にはなるんじゃないだろうか。

雨も滴るなんとやら。\scalebox{3}[1]{―}馬鹿馬鹿しくて恥ずかしくなってきた。

痛いぐらいに勢いが強いシャワーを浴びるように、
大粒の雨を受け続けていると、知らず知らずの内に体力は奪われていく。
走り始めてから五分ほどたったはずなのに、一向に目的の場所は姿を見せない。
わりかし体力には自信があったのだが、
今の状態だとたった十メートルほどを走っただけで、私の体はすぐに音を上げてしまう。
ヘトヘトに成りつつも、このまま雨に打たれ続けるわけにもいかず、
休むことも急ぐことも辛い現状に私は今日という日を恨んでしまう。

ああ、なんて災難な日なんだろう。
感慨深く\scalebox{3}[1]{―}実際にはそれっぽくしただけだが\scalebox{3}[1]{―}私は天を仰ぎ、
目に雨粒を受けてしまう。

私は恨もうにも恨めない自然の偉大さに、ただ辟易とするだけだった。\\

鐘の音に振り向く人々の視線を感じて、私は縮こまってしまう。
あんな小娘がなぜここに、と皆聞きたがっている。
だがこんなときにこそ、堂々と立ち振る舞うべきなのだ。
開いたアンティーク調の扉を抜けて、私は張り詰める空気を目一杯肺に貯め込む。

そしてわざとらしいほど大きな歩幅で、席に向かった。

独特な匂いの充満する店内の雰囲気は、故意さえ感じる程に古臭く、落ち着きのあるものだった。
そんな中を、恥ずかしいぐらい大股を開いて闊歩する私は、逆に不遜なのではないだろうか。
まあ、席に到達した今、それを後悔する必要もない。
私は目の前の人間に会釈して、長身の女の横に座ろうとした。

「遅いぞ、\ruby{玲華}{レイカ}」\\
結った長髪を\ruby{靡}{なび}かせながら、女は不満そうに、あからさまな声で唸った。
それに、私が引こうとした椅子にわざと足を絡めて、妨害してくる。\\
「冬姫がもっと早く時間設定してれば良かったんでしょ」\\
これは断じて私の責任ではない。主張は激しく、私に引くつもりはない。
ただ依頼者の眼前で、この先予測される痴態を晒せば、
信用など地に落ちて\scalebox{3}[1]{―}最悪地の底にのめり込む\scalebox{3}[1]{―}せっかくの収入の機会を失ってしまう。
ここは互いに譲歩して、私はひとまず席についた。

私の席の前に、依頼者の女性は座っていた。
肩をすぼめて、落ち着きのなさそうな振る舞いを続けている。
おそらく、私達の不仲な場面を目撃してしまって、どうすれば良いのか狼狽えているんだ。

断っておくが、私とこの女の関係は、決して悪いものではない。
時折あっちが、常識とずれているだけだ。
今日だって、待ち合わせに指定された時間は午後の一時で、私は学校だった。
わざわざ先生たちの目を盗んで抜け出してきたのに、非難とは散々な仕打ちだろう。

「はじめまして。私は\ruby{御巫冬姫}{みかなぎかずき}。こっちは\scalebox{3}[1]{―}」\\
「\ruby{九条玲華}{くじょうれいか}です。よろしくです」\\
私は頭を下げたが、冬姫は下げない。この女は誰にでもこんなスタンスで、
本人曰く、このほうが友好的に見えるだろうと豪語するが、どう見ても不躾な人間にしか見えない。
或いは、何事も斜に構えて他人とは違うと高をくくっている勘違い野郎、か。

何れにせよ、ここで深く掘り下げる話題ではない。
私は切り替えて、目の前の人間に興味を向ける。

小柄で、存在感の薄い人だ。
ショートカットとかなりの目の隈が相まって、暗い印象を与えがちな顔立ちをしている。
しかし整っている。
少し血色を明るくすれば、その薄幸さも洗い流して、そこそこなものにはできるだろう。

品評はここまでにして、私は彼女の言葉を待った。

「\ruby{工藤南}{くどうみなみ}です」\\
聞きづらい声だ。小さすぎる。
喉の半分が塞がっているのか。
無意識でも意識しているにせよ、どこか抑えている気がする。
隠し事、とまではいかないが、後ろめたさに支配されていることは確実だった。\\
「それで、工藤さん\scalebox{3}[1]{―}」冬姫は率直に聞くつもりだ
「\scalebox{3}[1]{―}今日はどのようなご用件で?」\\
工藤は口を開かない。黙ってばかりで、その目は酷く泳いでいる。
塵でも追いかけているのかと疑いたくなる。
無駄な時間を消費することに、私は苛立ちを覚える。

貧乏ゆすりを高速で繰り返す私の足を、冬姫はヒールの踵で踏んづけた。\\
「イタッ」\\
思わず声に出てしまったが、工藤には聞こえていないだろう。
耳元で囁かれる「おとなしくしていろ」の言葉通り、私はわざとらしく背中を伸ばし、
そのまま全身を固定した。

「あの、話しても、良いですか?」\\
「どうぞ、お構いなく」\\
「えっと、その、何から言ったらいいかわからないんですけど、その、
あの、変なことなんですけど、私\scalebox{3}[1]{―}赤ちゃんを取られたんです」\\
驚いた。彼女が経産婦だったなんて。
失礼かもしれないが、どう見ても子供を養えそうな人間ではなかった。\\
「誘拐、というわけですか」\\
誘拐とは大抵、『神隠し』のことだ。それが私達二人の、いや依頼者も含めての共通認識だ。
ありきたりではあるが、私達の仕事の大半を占めるもの。
だったら、今回もつまらないものだ。
これらの結末は、親子との感動の再会、惨たらしい別れ、ただの仲直り、夜逃げ、家出……。
原因が誘拐犯や家族喧嘩から、『悪い』魔法使いや家督相続問題に挿げ替えられるだけだ。

早々に失せた興味の欠落を埋めるように、私の手は遊び始める。
指を鳴らして、簡単なストレッチをする。

ああ、早く帰りたいな。
そう思っていた矢先、私の意識を強引に引き戻す会話が、二人の間に続いていた。\\
「いや、誘拐じゃないくて。あ、な、なんと言うか」\\
「落ち着いて。深呼吸でも」\\
工藤の口調は激しくなっていた。
冬姫の提案など端から聞く気もない。
溢れ出る言葉を、まだ辛うじて残る理性で選別しようと努力しているが、
時期決壊するのは明白だった。\\
「赤ちゃん、お腹の中に、中にいたんですよ。だから、そんなことあるはずないのに。
消えたんですよ。お腹から。子供、私の子供。どこに、どこに言ったのかわからなくて。
でも痛くもなくて。重くもない。\scalebox{3}[1]{―}おかしいですよね?」

私達は反応に困っていた。
工藤の言葉をそのままに受け取れば、彼女は妊娠中に胎児を失くしたという。
しかも一切の外傷も違和感もなく、自身の知らぬところ突如として喪失した。

特異な事例だ。冬姫もより慎重に事を運ぼうとする。
こういった場合、あらゆる可能性を求めるべきだ。
それが私達二人の教訓であり、彼女はそれに倣う。\\
「想像妊娠、というものをご存知ですか?」\\
「想像……」\\
「実際には妊娠していないにもかかわらず、
妊娠における兆候が現れる、一種の心身症なのですが\scalebox{3}[1]{―}」\\
「そんなはずない!」\\
机を叩く工藤。あからさまな怒り。
突然の剣幕に、私は思わず反応してしまう。\\
「私は、私には\scalebox{3}[1]{―}いたんです。絶対に、赤ちゃんがいた! 嘘じゃない。
産もうって決めてた。検査もした。つわりもあった。決めてたのに。私の子どもなのに。
どうして、どうして……」\\
ただ「どうして」を繰り返して、また並列に机をどんどんと叩き続ける。
工藤はもはや壊れたスピーカーだ。これ以上の対話は見込めない。
周囲の目も集まりつつある。このままいけば私達も店側も、誰も得しない。

冬姫と示し合わせて、私は工藤の肩を持ち上げた。
力なく垂れる工藤の体を、もはや引きずるように。
泣き止むのも待たずに、私は激しく上下する背中を抑えて、彼女を出口に誘導する。

外は相変わらずの雨で、分厚い雲に日光はほとんど遮られていた。
青みがかった影の中を進む。

途中、彼女の涙が、私の腕になすりつけられる。

気持ち悪い生ぬるさ。雨と同じ熱と、更に増した粘度を持った雫。
拭いたいが、手は塞がっている。

どうして彼女は、ここまで熱い涙を流すのだろうか。\\

\scalebox{3}[1]{―}もはや疑うことは出来ないだろうと、冬姫は言った。\\
「私は彼女を家に帰す。玲華、今日はここまでだ」\\
近くのパーキングに止められた、冬姫の車に工藤を載せて、今日の仕事は終了した。

\section*{\tt Interlude:Falling in the sky (1)}
{\gt
妊娠期間の長い、離巣性の動物と考えることができるヒトは、
しかし本来予想される出産時期よりも早く産まれ落ちる。

ポルトマンはこれを『生理的早産』と定義し、
生理的早産によって産まれた無能な新生児を『子宮外の胎児』と呼んだ。
}

\section{}
私の皮膚を焼く日射に音を上げて、私は日陰に逃げ落ちた。
ブロック塀にもたれかかって座り込む。
背の低い建物の並木道だ。
完全な住宅街で、ここには消費しかない。
だから、平日昼間の町中は驚くほど静かで、動く日陰を見失えば、
ここは静止した時間の中だと言われても、なんら疑いようがないだろう。
白い日差しの、中途半端に漂白されて、惜しいところまで純白に近づいた灰色の世界。

そろそろ出席日数を考えはじめないと。
先日の面談の続きを、まさか彼女の家で再開させるなんて。
しかも、ひと目を気にしてこんな時間帯を指定され、
また私は無断欠席する羽目になってしまった。

ただあのままで放置できるかというと、それは無理なことだろう。
人であるのならば、彼女の錯乱\scalebox{3}[1]{―}助けを求める最後の手段だ\scalebox{3}[1]{―}を無視して、何食わぬ顔でどうやって日常に戻れるだろうか。
私は無理だ。

その後の話を冬姫から聞いた。
工藤はずっと塞ぎ込んで、呼吸すら徒に疲れを誘うだけだったという。

彼女は相当な精神力を消費して、あの場にまで出向いていた。
工藤はここ最近、自らに降り掛かった何らかの災難によって、
もはや日常生活を正常に営めない状態にあるという。
一日中引きこもって、食べるものもなく。
餓死の寸前に近所の住人に助けられたという。
それがなければ今頃は\scalebox{3}[1]{―}。

正直、初対面はただのヒステリー女かと思っていたが、
その認識は改めなくてはならない。

この熱さが落ち着いたら行こう。

待ち合わせは正午きっかり。

そろそろ立ち上がろう。

私はそう思って腰を上げようとする。
手に食い込むアスファルトの痛み。
体重がさらに境界を変形させて、凸凹な痕を記憶する。
だがその一連の流れを断ち切って、私に話しかける誰かが目の前にいた。

「手伝いましょうか?」\\
女性の声だった。

一瞬顔が地面に向いた間に、おそらく彼女は私の前を通りかかったのだろう。
そして親切に手を貸そうとしてくれている。
だが見知らぬ人間に話しかけられて、驚かない人間はいない。
それが特に美麗な人間なら尚更にだ。

ふと顔を上げて見ると、私は不意を疲れたように尻もちをついてしまった。\\

「ああ、ごめんなさい。そんなつもりはなかったの」\\
慌てて私を起こそうとする女性。
しゃがんで私の背を押してくれる。\\
「ありがとうございます」\\
「いえいえ」\\
微笑む顔に、私は目を奪われた。
完成されている。
強制的にそう感じさせられる彼女の顔は、
心理の奥深くに刻まれた、普遍的なものに依拠しているはずだ。
無条件的な好意は、まるで子どもだ。
だが長い濡鴉の髪が、彼女の幼稚さを引き締めて、彼女を面妖な女性に昇華させている。
白いワンポースとつばの広い帽子も、私より少し大きな身長によく似合っている。
ただ気になるのは、彼女の腕や胴、首周りで、全体的に長く痩せ細っている。
例外はその胸だろうが、それでもこの日差しの中の真っ白すぎる肌は、不健康さを醸し出していた。
夏至の雪だるま。
それが私の彼女に対する第一印象だ。

その逆を計るとどうだろう。
私の格好は黒地のシャツに白のショートパンツ。
着飾るつもりもないし、最低限のもので十分だ。
素朴な人間だろうか。
少なくとも服は大して判断材料にはならないだろう。
だとすれば顔だ。
髪型はアシンメトリーで、左耳を隠すようなショートボブだ。
自分で言うのもなんだが、顔はそこまで悪くはない。
整っているが、これと言って特徴もない。

彼女は私をどう思うのだろうか。
この思案は、私の悪い癖の一つだ。

願わくばこのまま別れて、また互いの道に戻りたいのだが。

\scalebox{3}[1]{―}どうやら彼女は、それを望んでいないらしい。\\
「なんだか、こんなこと言うのも変だけれど、
良かったら、お茶でもご一緒にいかがかしら?」\\
おいおい、かなり踏み込んできたじゃないか。\\
「ああ、それは……良いですね」\\
本心は断りたがっているが、ここで拒否できるほどの勇気もない。

それに、私の返事にうきうきな彼女を見ても、そこまで悪くはなかった。

\section{}
時計を忘れていたことに、いまさら気がついた。
これでどうやって、時間通りに待ち合わせをしようというのだろうか。

彼女と出会った後、私達二人は少し歩いて、駅前のカフェに入った。
そこでやっと、私は正確な時間を知った。
現在時刻ただいま十時十七分。
時間的な余裕は、浪費できるほどある。
それに彼女がここを選んでくれてよかった。
たしかこの駅の近くに、工藤の住むマンションがあったはずだ。

こうなると、心に落ち着きが戻ってくる。
初対面の人間に対しても、それなりの態度を取れるようになる。
調子に乗るのだ。これも私の悪い癖、というよりか臆病な性格の表れだろう。
その証拠に、向かい合う彼女とは目も合わせられない。
澄んだ黒目は、永遠と深い孔を連想させる。
意識の蟻地獄。彼女はただそこに居ればいい。
後は、あの清廉さに抗えない私のような餌が、勝手に転がり込んでくる。
どう考えても、彼女の体を構成するパーツ一つ一つの引力は、常人のそれではない。
なのに、それら全てを統合した彼女自体は、性質が真逆で儚い陽炎だ。
質量を持たない影。本当に不思議な人間で、私はいつの間にか惹かれている。

仕方なく私は目を建物に向ける。まあ、ありきたりなデザインで、いかにもな建築だった。
ただ前面がガラス張りになっていて、外の景色がよく見える。
逆に言えば、店内のプライバシーはないようなものだった。

「何か好きなものでも選んで」\\
メニューを差し出され、私は自然と受け取る。
いろいろあるが、どれも同じように見える。
結局私はコーラを注文しようと思った。
後は、フレンチトーストでも食べてみようか。
ここまで決まってやっと、私は遠慮するというプロセスをすっぽかしたことに気がついた。\\
「その、お金は別に自分で払います……よ?」\\
「そんな事言わないで。私がお願いして付き合ってもらってるんだから。
これは、私の気持ち。遠慮しないで」\\
今どきこんなにも純粋な心を持つ人間が居るなんて、私は勝手に感動する。
純粋過ぎて逆に心配になってしまう。
私がもし悪い人間だったら、もう何万円かくすねてやろうと下心を丸出しにしているだろう。

「じゃあ、これで」\\
私はメニューの写真を指さして、店員に注文した。
かしこまりました、と定型文を述べて次に進む。
彼女は直前まで何も決まっていなかったようだ。\\
「あら、美味しそう。私もそれ頼んでいいかしら?」\\
「どうぞどうぞ」\\
「それでは二つ、でよろしいでしょうか?」\\
「はい」\\
「以上でよろしいでしょうか」\\
「はい、以上で」\\
「ご注文ご確認ください\scalebox{3}[1]{―}」

それから私達は料理が運ばれるまでの間、互いの虫食いパズルを埋めようと努力し始めた。\\

「失礼かもしれないけれど、どちらからいらっしゃったの? 
あまり見覚えのない顔だったから、つい、話しかけちゃって」\\
彼女の質問に、私はありのまま答えた。\\
「まあ隣の隣街、ぐらいですかねえ」\\
「やっぱり。何だか雰囲気が違うもの」\\
私は異物だと言うことだ。
だがわからなくもない。
自分の近所を通る人間の、馴染んでいるか不馴れであるかは、かなりの確率で部外者を選別できる。
経験則とはこういうものだろう。\\
「やっぱり分かるものなんですね」\\
「\scalebox{3}[1]{―}そう。この町の人は、みんな貴女みたいに大胆じゃないから」\\
確かに、道端で座り込む人間はそういないだろう。\\
「あれは、ちょっと疲れてて」\\
「でも地面に座るなんて。私には絶対できない」\\
それは見下しているのではなく、本心からそう言っているのだろう。
一種の羨ましさすら感じる口調だ。\\
「小さい頃とかは、よく公園から裸足で家に帰ったり、泥だらけで家の中走り回ってました」\\
「あら、活発。良いわね。羨ましい」\\
「羨むようなことでもないですよ、全然」\\
「私ね、体が弱くって、外に出られなかったの。今でも、歩くので精一杯」\\
なんとなく察していたことだった。だから、特に気遣うこともないし、逆にそのほうが失礼だろう。\\
「\scalebox{3}[1]{―}貴女みたいに、元気だったらっていつも思ってた」\\
「走り回って怪我するよりはマシですよ」\\
「ふふ、そうね。痛いのは御免ね」\\
あまりその意図はなかったのだが、彼女は笑ってくれた。
彼女は身の弱さを嘆いているが、しかし極端に不自由といったわけでもなさそうだ。
むしろ解放されたように、生き生きとしている。\\

「お待たせいたしました、こちら\scalebox{3}[1]{―}」\\
フレンチトーストと飲み物が運ばれた。
写真で見るより、案外いい色味だった。
それに匂いもいい。バターの匂いだ。本能に訴える油脂の誘惑。\\
「美味しそうね。貴女と同じにして正解だった」\\
「ありがとうございます」\\
なんだか褒められた気分になったから、自然とそう出た。
母親みたいだ。母性の片鱗。やはり女性は多かれ少なかれそういったものを持っている。
特に彼女のものは強い。私は今すぐナイフで音を立てて、注意されたい欲求が高まる。
「めっ」という仕草は、きっと彼女によく似合う。
だがそれは妄想に抑えておくべきだ。いくらなんでも、気持ち悪すぎる。\\
「どうしたの、食べないの?」\\
ほら、言った傍からこれだ。\\
「食べますよ。食べます」\\
いただきます。と教えられたての子どものように、私は元気よく言った。\\
「\scalebox{3}[1]{―}美味しい」\\
期待以上だった。\\



\end{document}